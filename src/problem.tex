\section{Problem Statement}\label{sec:problem}

The first obvious thing which a thesis proposal should contain is a statement of the problem to be considered, in both specific and general terms. The specific statement must deal with the very specific issues in which the candidate is interested, for example, the optimization of tables of LAIR parsers. The general statement should relate the problem to the larger context of the science and show why it is worth solving. The problem statement in the thesis proposal should be directed to an audience of intelligent scientists who have no specific interest in the problem but who are interested in knowing what the candidate is doing. It should not be directed to the candidate's supervisors and/or to people with similar research interests.

To prepare the proposal for their benefit is to make a very common mistake. Such a proposal is filled with jargon which is private to that local group. It fails to state important constraints and frequently does not provide enough background. Sometimes the candidate assumes that his supervisors know as much about the specific area of the thesis as he does something which makes it difficult for the department and the examiners to evaluate the research on its merits. The candidate is then exposed to the very real danger that he and supervisors may have been working very happily in their own microcosm, only to find that at the end of three years he has no results which justify a Ph.D. degree.




%%\note{example}


%%\note{principles}


%%\note{issue}


%%\note{assumptions}


%%\note{objectives/metrics}


%%\note{theses}
