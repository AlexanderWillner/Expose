\section{Own Approach and Considered Methods}\label{sec:ideas}

It is hard enough to schedule 'invention' when one has some good ideas for solving a problem. It is almost impossible when he does not. Thus the Ph.D. student, who is working to a tight and very emotionally constraining timetable, needs to have some insight, some ideas, some preliminary results before he commits himself to discover more. These should be described in the third section of the thesis proposal. If he has none of significance, then his proposal is premature. For he would have no indication that the problem can capture his attention for as long as it takes to solve it an write the thesis. He would have no assurance that he is heading in the right direction, that he is capable of finding a solution.

By implication, then, the candidate must have done some successful work in the area, perhaps in collaboration with others, before the thesis proposal. This may be something like the discovery of an interesting algorithm, representation, or relation while working on one of his pre-thesis projects. He recognized this as a tip of the iceberg, the introduction to a new problem area which eventually becomes his thesis research. For example, a student simulating a well-know paging algorithm stumbles across a phenomenon quite different from that which was expected or generally accepted. This result and his subsequent explanation for it form the basis of his thesis proposal and thesis research in memory management. They form the seed of the methods which he develops to specify and solve his problem. Without such results, a plan to investigate the area would have seemed like hot air, and his efforts would have lacked direction. But with them, the success of his research is assured and the timely completion of his thesis is much more likely.

A common situation occurs when a student proposes what seems to be a good problem to investigate, involving brand new broad, general models or theories. But when he is pressed, he has only some ideas about a very small, special case or example. He might not even have explored these ideas fully because he regards that example as uninteresting in the context of the overall problem and those ideas as having no apparent generalization. Some students will be able to discover the necessary general ideas, develop them and defend them. But such theses are few and far between, and their authors are typically awarded Nobel prizes and other very high distinctions. Ordinary mortals with good first class honours degrees have no such luck and often get stuck, unable to find any other examples, applications or ideas which are substantially different from the ones they know already.

At this point, it is time to go back and look at the problem statement again. As often as not, that "uninteresting" example may be the foundation for an interesting and valuable thesis problem in its own right. If so, it is probably a better investment of the candidate's energy to solve it, finish his thesis, and then devote his life's work to the general problem in a more relaxed fashion.

Summary:

\begin{itemize}
  \item Your ideas and insights for solving the problem and any preliminary results he may have obtained
  \item You need to have some insights first - otherwise this proposal is premature.
  \item Which methods do exist? What are they assets and drawbacks?
  \item The way in which you are going to solve the issue (mathematical theory, model, simulation tool, ...)
  \item What, how and why to use a specific method for which sub problem?
  \item In general you'll not solve a general problem but an interesting example. 
  \item Show theses and hypotheses .
  \item Describe the theoretical basis. 
\end{itemize}
