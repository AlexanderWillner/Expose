% PhD: Reference to and comments upon relevant work by others on the same or
%      similar problems
% PhD: Do *not* only reference work from your department!
% PhD: Be aware of the "Not Invented Here" issue
\section{Related Work}\label{sec:relatedwork}


In order to present the problem to the wider audience, and in order to justify proceeding with the work, it is necessary for the candidate to present the background to the problem and to survey related work by others. This is the second component of a thesis proposal; and in some cases, it may be included directly in the thesis. It may take any of several forms-for example, annotated bibliography or a comprehensive summary, explanation, and analysis of existing results. It may be necessary or desirable for the candidate to include his own critical comments. For example, if the thesis is to present a new technique for solving a class of numerical problems, then this section of the proposal should review existing techniques and analyze their inadequacies.

This summary/survey/overview is not without its traps. If most of the references cited and most of the work mentioned are from within the candidate's own department (or in one other department with whom we are very "chummy") then there are serious grounds for questioning his breadth of knowledge and background for pursuing his problem. The danger is that people who limit their horizons to their own local environments produce very inbred research, narrow attitudes, and unacceptable theses. They tend to reinvent ideas already known elsewhere; they fail to apply techniques which could simplify their problems considerably; they often attach too much importance to minor results and do not recognize major ones worth reporting; and they write incomprehensible theses and papers which make no effective contribution to knowledge. In inbred environments, the work of other organizations is often dismissed as irrelevant or unimportant characteristic of a disease called NIH (Not Invented Here). It is extremely important for the thesis proposal to indicate that the candidate knows about the complete work.

The literature survey is a broad and shallow account of the field, which helps
to place the contribution of the paper in context. It is part of the
motivation of the paper, because it helps to identify the gap that this work
is trying to fill, and explain why it is important to fill this gap. Rather 
than a list of disconnected accounts of other people's work, you should try to 
organise it into a story: What are the rival approaches? What are the 
drawbacks of each?
How has the battle between different approaches progressed? What are the 
major outstanding problems? (This is where you come in.)

Summary: 

\begin{itemize}
  \item Show the historical development of the subject
  \item Reference to and comments upon relevant work by others on the same or similar problems
  \item Reference only a few key papers - a summary of the gap in the field you have identified.
  \item Was the same issue examined before (other PhD theses)?
  \item What are the most important stances in the literature?
\end{itemize}
