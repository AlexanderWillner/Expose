% PhD. A rough outline of the thesis itself
% PhD. In terms of expected solutions to the problem
\section{Thesis Outline}\label{sec:outline}
Finally, it is always useful when doing research to keep in mind how it is to be reported, what issues will be emphasized, and what will be de-emphasized. Thus, the thesis proposal should contain a rough outline of the thesis itself, preferably in terms of the expected solution to the problem. This will have at least a small impact on the shape of the research, and it will provide a set of good guidelines when the candidate decides that it is time to "write it all up".

A rough outline of the thesis itself, in terms of expected solutions to the problem, is given below:

\begin{itemize}
   \item[\textbf{Introduction} \textit{(4 p.)}:]~\\
   Introduction to lead the reader into the research area and create an awareness of the problem and its importance. Give a brief overview of the context of work and expose the own contributions to the research. Sketch the most important validation results and give an outline of the thesis.
   \item[\textbf{Related Work and Research Gaps} \textit{(40 p.)}:]~\\
   Essential fundamentals for the reader. Identification of the research gap by referencing to and comments upon relevant work by others on the same or similar problems. The most important stances in the literature will be exposed and a historical development of the subject will be given.
   \item[\textbf{Assumptions, Objectives and Scope} \textit{(20 p.)}:]~\\
   Clear definition and dissociation of the thesis' scope. Description of the assumed presuppositions and o\-ver\-view of the limits of the research. Explanation of the objectives in a more formal manner.
   \item[\textbf{Main Part} \textit{(40 p.)}:]
   Assets and drawbacks of existing strategies to solve the problem are discussed and the own ideas, insights, and ways for solving are defined and implemented; in terms of an analytical model and simulation environment.
   \item[\textbf{Influence on Scheduling Performance} \textit{(30 p.)}:]~\\
   Exhaustive analysis of the own approach and comparison with other existing solutions. Interpretation of the results and recapitulation of the achieved contribution to the research; if possible based on data from existing applications.
   \item[\textbf{Conclusions, Discussions, Future Work} \textit{(4 p.)}:]~\\
   Synopsis of the work done in the thesis. Identification of the key contributions and discussion about further research in this area.
\end{itemize}
