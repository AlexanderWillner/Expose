\section{Expected Solution}\label{sec:shape}

The most important part of the thesis proposal is a statement of what kind of solution to the problem is expected, i.e., a characterization of the stopping condition of the project. This, more than anything else, will help the candidate estimate the value of his efforts to separate the chaff from the wheat, to allocate his time. Without such a characterization, the candidate has no good way of knowing when to stop and submit. He cannot measure how far towards his goal of a Ph.D. degree he has progressed. He might even discover a satisfactory solution to his problem and not perceive that he has. With a characterization, he will know where he stands during his research, and he will be able to argue convincingly at the appropriate time that he has done what he set out to do.

Occasionally, a research student will say, "I know precisely what problem I want to solve. I have no idea of what the solution will be, but I will certainly recognize it when I've got it. After all, this is research. So how can I possibly give a characterization of the solution beforehand?" That is, he thinks he is an exception, but if he cannot characterize his expected solution, how can he recognize it? More likely, he has not specified his problem sufficiently precisely, or he has not yet done enough preliminary work and obtained some preliminary results in the area of the problem. In either case, he must do more legwork before presenting his thesis proposal. Sometimes it is easy to characterize the solution, particularly in the light of preliminary results. For example, a candidate developing a new analytical model to describe message traffic among communicating machines would expect to prove some theorems about the model, validate it empirically against some existing systems, construct some algorithms based on it for calculating the performance of similar systems with different parameters, and argue by example that they are useful in the design and understanding of future systems. At other times, it is much harder to be so specific about a stopping condition. It may also be necessary to change it as the research progresses. However, a moving target is better than no target at all (providing that it is not moving so fast that the candidate cannot catch it.)

Summary:

\begin{itemize}
  \item Any preliminary results that may have obtained;
  \item Short summary of results that are expected from the research.
  \item Characterization of what sort of solution is being sought;
  \item Define a criteria for success -- establish exactly the criteria by
  which you intend to judge the success of the work.
  \item A statement or characterization of what kind of solution is being sought
  \item What type of solution to the problem is expected? Stopping condition.   
\end{itemize}
