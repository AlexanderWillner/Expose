\section{Introduction}\label{sec:introduction}

% ------------------------------------------------------------------------------
% The introduction should lead the reader to the work you've done and you're
% planning to do. You should
% start with general statements and then get close to the core (zoom in). Every
% computer scientist should be able to understand what is paper is about. The
% introduction should follow the same structure as the abstract; but this time
% you should spend one paragraph for each item.
%
% (1) What is the problem? Motivation: broadly, what is the problem area,
% why it is important? Open up the subject (the subject will be electromagnetic
% fields in cylindrical dielectric geometrics, adaptive arrays in packet radio,
% or whatever.)
% Introduce problem, outline the solution; the statement of the problem should
% include a clear statement why the problem is important (or interesting). In
% the case of a conference, make sure to cite the work of the PC co-chairs and
% as many other PC members as are remotely plausible, as well as from anything
% relevant from the previous two proceedings. In the case of a journal or
% magazine, cite anything relevant from last 2-3 years or so volumes. Avoid
% stock and cliche phrases such as "recent  advances in XYZ" or anything
% alluding to the growth of the Internet. Be sure that the introduction lets
% the reader know what this paper is about, not just how important your general
% area of research is. Readers won't stick with you for three pages to find out
% what you are talking about.
% The introduction must motivate your work by pinpointing the problem you are
% addressing and then give an overview of your approach and/ or contributions
% (and perhaps even a general description of your results). In this way, the
% intro sets up my expectations for the rest of your paper -- it provides the
% context, and a preview. Repeating the abstract in the introduction is a waste
% of space.
%
% A statement of the problem and why it should be solved
% Exactly focus and identify the problem
% Identify both: the particular problem and the foundational/global issue (area)
% Identify the fundamental theoretical and methodical principles
% Show the context.
% Why should someone care?
% Identify the rough objectives
% Exactly identify the assumptions
% Formulate theses
% Why should the issue be solved?
% Are there real examples?
% ------------------------------------------------------------------------------
The introduction should lead the reader to the work you've done and you're
planning to do. You should start with general statements and then get close to the core (zoom in). Every computer scientist should be able to understand what is paper is about. The introduction should follow the same structure as the abstract; but this time you should spend one paragraph for each item.

A statement of the problem and why it should be solved.
Exactly focus and identify the problem.
Identify both: the particular problem and the foundational/global issue (area).
Identify the fundamental theoretical and methodical principles.
Show the context.
Why should someone care?
Identify the rough objectives.
Exactly identify the assumptions.
Formulate theses.
Why should the issue be solved?
Are there real examples?

Motivation: broadly, what is the problem area, why it is important? Open up the subject (the subject will be electromagnetic
fields in cylindrical dielectric geometrics, adaptive arrays in packet radio,
or whatever.)
Introduce problem, outline the solution; the statement of the problem should
include a clear statement why the problem is important (or interesting). Avoid
stock and cliche phrases such as "recent  advances in XYZ" or anything
alluding to the growth of the Internet. Be sure that the introduction lets
the reader know what this paper is about, not just how important your general
area of research is. Readers won't stick with you for three pages to find out
what you are talking about.
The introduction must motivate your work by pinpointing the problem you are
addressing and then give an overview of your approach and/ or contributions
(and perhaps even a general description of your results). In this way, the
intro sets up my expectations for the rest of your paper -- it provides the
context, and a preview. Repeating the abstract in the introduction is a waste
of space.


% ------------------------------------------------------------------------------
% % (2) Why is it interesting and important? Why is it hard? (e.g., why do
% % naive approaches fail?) Narrow down: what is problem you specifically
% % consider? Describe the problem addressed in this paper.
% % 
% % Reference to and comments upon relevant work by others on the same or
% % similar problems
% ------------------------------------------------------------------------------
Why is it interesting and important? Why is it hard? (e.g., why do
naive approaches fail?) Narrow down: what is problem you specifically consider? Describe the problem addressed in this paper.

% ------------------------------------------------------------------------------
% % (3) Survey past work relevant to this paper. Why hasn't it been solved
% % before (related work)? Or, what's wrong with previous proposed solutions?
% % How does mine differ?
% % 
% % Reference to and comments upon relevant work by others on the same or
% % similar problems.
% ------------------------------------------------------------------------------
Survey past work relevant to this paper. Why hasn't it been solved before (related work)? Or, what's wrong with previous proposed solutions? How does mine differ? Reference to and comments upon relevant work by others on the same or similar problems.


% ------------------------------------------------------------------------------
% % (4) Describe the assumptions made in general terms, and state what results
% % have been obtained. This gives the reader an initial overview of what
% % problem is addressed in the paper and what has been achieved.
% % 
% % The candidate's ideas and insights for solving the problem and any
% % preliminary results he may have obtained
% ------------------------------------------------------------------------------
Describe the assumptions made in general terms, and state what results have been obtained. This gives the reader an initial overview of what problem is addressed in the paper and what has been achieved.
The candidate's ideas and insights for solving the problem and any preliminary results he may have obtained.

% ------------------------------------------------------------------------------
% % (5) What are the key components of my approach and results? Also include any
% % specific limitations. ``In the paper, we ...": most crucial paragraph, tell
% % your elevator pitch: How is it different/better/relates to other work?
% ------------------------------------------------------------------------------
What are the key components of my approach and results? Also include any specific limitations. ``In the thesis, we ...": most crucial paragraph, tell your elevator pitch: How is it different/better/relates to other work?
What are the key components of my approach and results? Also include any
specific limitations.


% ------------------------------------------------------------------------------
% % (6) A plan of action for the remainder of the research
% ------------------------------------------------------------------------------
A plan of action for the remainder of the research.

% ------------------------------------------------------------------------------
% % (7) The second last item must start with "The main contributions of this thesis will be..." to help the reviewer to get the scientific surplus value between all the motivation and basics.
% ------------------------------------------------------------------------------
The main contributions of this thesis will be\ldots (to help the reviewer to get the scientific surplus value between all the motivation and basics).

% ------------------------------------------------------------------------------
% % (8) ``The remainder of this paper is structured as follows...''.
% ------------------------------------------------------------------------------
%\note{outline}
The remainder of this proposal is structured as follows (based on \cite{Lauer:1975gf}): In
\sectionname~\ref{sec:problem} the particular and the fundamental problem is
identified in detail. Subsequently, the historical development of the subject,
relevant work by others, the most important stances in the literature, and a
summary of the gap in the field are given in
\sectionname~\ref{sec:relatedwork}. Based on this knowledge in \sectionname~\ref{sec:ideas} booth the
approaches and considered methods for solving the problem are discussed. The
preliminary results, a short summary of results that are expected from the
research and the characterization of what sort of solution is being thought are
given in \sectionname~\ref{sec:shape}. Furthermore, in
\sectionname~\ref{sec:actions} a plan of actions for the remainder of the
research, the major milestones, and the planned publications are given. Finally,
in \sectionname~\ref{sec:outline} a rough outline of the thesis itself, in
terms of the expected solution to the problem, is shown.

% ------------------------------------------------------------------------------
% In German:
% ------------------------------------------------------------------------------
% Der Rest des Exposés ist, basierend auf \cite{Lauer:1975gf}, wie folgt
% gegliedert:
% In \sectionname~\ref{sec:problem} wird die zugrundeliegende Problemstellung
% und ihre Relevanz beschrieben. 
% Anschließend wird in \sectionname~\ref{sec:relatedwork} der aktuelle Stand der
% Forschung aufgezeigt und relevante Veröffentlichungen diskutiert.
% In diesem Zusammenhang werden offene Fragestellungen identifiziert und die
% eigene Arbeit in das Forschungsumfeld eingebettet.
% Darauf aufbauend ist in \sectionname~\ref{sec:ideas} der eigene Lösungsansatz
% beschrieben und erste vorläufige Ergebnisse werden präsentiert. Dies schafft
% die Grundlage für eine Charakterisierung erwünschter Ergebnisse in
% \sectionname~\ref{sec:shape}, um das Ziel der Arbeit zu konkretisieren.
% Das weitere geplante Vorgehen erläutert
% \sectionname~\ref{sec:actions} und ein entsprechender Zeitplan ist dargelegt.
% Abschließend wird in \sectionname~\ref{sec:outline} eine grobe
% Übersicht der geplanten schriftlichen Ausarbeitung widergegeben.



%\note{academic field}
%\note{the challenge}
%\note{deficiencies}
%\note{objective/assumptions}
%\note{validation}
