\section{Plan of Work}\label{sec:actions}
The first two points which a thesis proposal should address are almost, but not quite, afterthoughts. After the candidate knows what he wants to do, has some background to allow him to do it, has done a little bit, and has some idea where it will take him, he had better draw up a plan of action. This section of the thesis proposal is like a road map and timetable of how he will travel during the remainder of his research. If it is carefully and realistically prepared, it will expose to him any hazard of trying to do more than he reasonably can before he runs out of steam. Obviously this plan, like everything else in the proposal, is subject to change as new results are obtained and new ideas gained. But some plan is better than no plan.

So this is a plan of action for the remainder of the research - a roadmap / a timetable of how to travel through the research. This is just an example:

\begin{itemize}
 \item[\textbf{Milestone 1} \textit{(06/2011)}:]~\\
 Final thesis topic and content definition. To validate the planned strategy one or more talks are given.
 \item[\textbf{Milestone 2} \textit{(09/2011)}:]~\\
 Exhaustive analysis of the related work and the research gap. Results will be discussed within an introductory talk and published in form of an overview paper.
 \item[\textbf{Milestone 3} \textit{(01/2012)}:]~\\
 Detailed specification of the assumptions, objectives and scope of the own approach. First model and scheduling algorithms with initial evaluation results. The outcomes are presented in the second intermediate talks and published.
 \item[\textbf{Milestone 4} \textit{(04/2012)}:]~\\
 Sophisticated model and evaluation results based on real application data. A third intermediate talk is given and the results are published.
 \item[\textbf{Milestone 5} \textit{(09/2012)}:]~\\
 As well all open issues as all odds and ends are identified and closed. A final talk presents the key contributions.
\end{itemize}
